\documentclass{article} % For LaTeX2e
\usepackage{nips15submit_e,times}
\usepackage{hyperref}
\usepackage{url}
\usepackage{lineno}
\usepackage{graphicx}
\usepackage{amsmath}
\usepackage[all]{hypcap} 
%\linenumbers% Uncomment for line numbers



\title{Big Data, Big Problems--Parallelizing Kmeans}


\author{
Reuben K. McCreanor \\
Department of Statistics\\
Duke University\\
Durham, NC 27708 \\
\texttt{reuben.mccreanor@duke.edu} \\
\And
Wei (Emily) Shao\\
Department of Statistics\\
Duke University\\
Durham, NC 27708 \\
\texttt{wei.shao@duke.edu} \\
}

\hypersetup{
    colorlinks=true,
    linkcolor=blue,
    filecolor=magenta,      
    urlcolor=cyan,
    pdftitle={Sharelatex Example},
    bookmarks=true,
    pdfpagemode=FullScreen,
}


\newcommand{\fix}{\marginpar{FIX}}
\newcommand{\new}{\marginpar{NEW}}

\nipsfinalcopy 

\begin{document}

\maketitle

\begin{abstract}
K-means is a data-partitioning algorithm that separates n observations into k partitions. Due to its simplicity of implementation, interpretability of its results, and its ability to categorize data based on desired features, it remains one of the most popular algorithms in fields of statistics and machine learning. However, the key issue with k-means comes from its efficiency and scalability. As the size of datasets continues to increase, implementing k-means on data of any magnitude becomes computationally infeasible. A recently proposed variation of k-means, k-means++, provides a robust method of selecting the initial centers, essentially giving an optimal solution. However, due to the number of passes over the data required in a naive  implementation, even clusters a million data points into 100 partitions would be exceedingly slow. In order to combat this, a parallelized version of k-means++ is proposed, k-means $||$. This version uses a sampling factor $\ell$ to dramatically reduce the number of passes while still arriving at an equivalent solution of partitions. This paper will implement k-means++ and k-means$||$ in both sequential and parallel setting and compare the results both in terms of efficiency and equivalency of partition arrangements.
 
\end{abstract}

\clearpage
\newpage

\section{Introduction}
\label{headings}

\section{Overview of the Algorithms}
\label{headings}

\subsection{K-means++}

\subsection{K-means$||$}

\section{Implementation}
\label{headings}

\subsection{Sudo Code}

\subsection{Data Simulation}

\subsection{Testing}

\section{Optimization}
\label{headings}

\subsection{Numba}

\subsection{Cython}

\subsection{MapReduce and Hadoop}

\section{Results}
\label{headings}

\subsection{Comparison of Numerical Results}

\subsection{Comparison Clustering Arrangements}

\subsection{Comparison of Efficiency}

\section{Conclusions}
\label{headings}

\clearpage
\newpage

\subsubsection*{References}

Bahmani, Bahman, Benjamin Moseley, Andrea Vattani, Ravi Kumar, and Sergei Vassilvitskii. \textit{Scalable K-means.} Proc. VLDB Endow. Proceedings of the VLDB Endowment 5.7 (2012): 622-33. Web.


\clearpage
\newpage

\section{Appendix}
\label{headings}



\end{document}


